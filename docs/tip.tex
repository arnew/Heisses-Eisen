% vim:spell:spelllang=en_us
\section{mechanical tip properties}
Weller WMRP tip "RT 3" 1,3x0,4mm
approx 10cm
tip, heater, sensor, grip, 3,5mm jack
nominal 


\section{thermal element}


\subsection{common element types}
type K -> tried AD8495 AR: $5mV/\deg C$ output
other types: very unlikely


\subsection{custom measurement amplifier}
to accurately measure the thermocouple voltage and to record the tips characteristics, an amplifier has to satisfy the following conditions:
- Very high input resistance (thus very low input current) because of the t/cs low impedance
- very low offset to be able to amplify signals in the microvolt-range
- rail-to-rail input and output, because of low input voltage and 5V supply
- high linearity (low gain error)

AD 8552 AR
provides:
 - $1\,\mu V$ offset
 - $0,005\,\mu V/\deg C$ drift


With the AD 8552 a simple non-inverting amp was constructed. Gain was trimmed to $g=400$.

Previously constructed AD8495 circuit was used to provide reference temperature information.

WMRP Tip, reference type K thermocouple were closely thermally coupled to another soldering iron. An Atmel Atmega8 MCU with an 10bit ADC was used to record data.

First results showed a linear dependence of thermocouple voltage and temperature in a range of 150-250 $\deg C$ with a coefficient of about $16\,\mu V/K$.


\section{heating element}
